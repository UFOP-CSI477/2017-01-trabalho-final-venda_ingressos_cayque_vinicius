\documentclass[10pt,a4paper,article]{abntex2}
\usepackage[utf8]{inputenc}
\usepackage{amsmath}
\usepackage{amsfonts}
\usepackage{amssymb}
\usepackage{url}

% Nome dos autores do trabalho
\title{CSI477-2017-01 -- Proposta de Trabalho Final}
\author{Grupo: Cayque Domingues dos Santos \& Vinicius Arantes Mangela}
\begin{document}

	\maketitle

	% Descrever um resumo sobre o trabalho.
	\begin{abstract}
		O objetivo deste documento é apresentar uma proposta para o trabalho a ser desenvolvido na disciplina CSI477 -- Sistemas WEB I. É uma breve descrição sobre o tema que será abordado, bem como o escopo, as restrições e demais questões pertinentes ao contexto.
	\end{abstract}

	% Apresentar o tema.
	\section{Tema}

		O trabalho final tem como tema o desenvolvimento de um site para venda de ingressos de cinema.

	% Descrever e limitar o escopo da aplicação.
	\section{Escopo}

		Este projeto terá como objetivo, desenvolver uma plataforma onde donos de cinemas podem vender ingressos para seus clientes, bem como disponibilizar um ambiente de compra para os clientes. O site irá trabalhar com dois tipos de clientes: físicos e jurídicos, sendo o primeiro quem compra os ingressos e o segundo quem os disponibiliza. A diferenciação dos tipos de clientes será feita no momento do cadastro (CPF ou CNPJ), cada cliente terá uma visão diferente do site após o seu login, consequentemente podem realizar tarefas distintas dentro da plataforma, de maneira geral, um cliente jurídico pode cadastrar os seus cinemas, suas sessões e seus filmes, e o cliente físico cadastra os seus dados e suas preferências. 

	% Apresentar restrições de funcionalidades e de escopo.
	\section{Restrições}

		Neste trabalho não será considerado os meios de pagamento, ou seja, o cliente físico faz uma espécie de reserva na sessão do filme que deseja, então é gerado um \textit{ticket}, e o pagamento e feito na própria bilheteria do cinema, momentos antes do inicio do filme.

  % Construir alguns protótipos para a aplicação, disponibilizá-los no Github e descrever o que foi considerado.
	\section{Protótipos}
		  Protótipos para as páginas (descrever quais páginas) foram elaborados, e podem ser encontrados em...

	% Incluir o link do repositório
	\section{Repositório}

		O trabalho final terá como repositório principal o seguinte endereço: \url{https://github.com/UFOP-CSI477/2017-01-trabalho-final-venda_ingressos_cayque_vinicius}.

	% Referências podem ser incluídas, caso necessário.
	\section{Referências}
	Referências podem ser incluídas, caso necessário. Utilizar o padrão ABNT.

\end{document}
